\tikzset{
	data/.style={
		draw,
		rectangle split,
		rectangle split parts=2,
		text centered,
	},
	data+/.style={
		data,
		rectangle split every empty part={},% resets empty-part macro (explanation below)
		rectangle split empty part width=\widthof{#1},
		rectangle split empty part height=\heightof{#1},
		rectangle split empty part depth=\depthof{#1},
	},
}


\begin{tikzpicture}[background rectangle/.style={fill=green!15, rounded
	corners, draw=green!50}, show background rectangle, split/.style={rectangle split, rectangle split horizontal, rectangle split parts=#1, draw},splitv/.style={rectangle split, rectangle split parts=#1, draw}]



\node (R) {Root};
\node[split=1, above=0.25cm of R, fill=black!60] (A) {};


\node [data, right=1.0cm of A, fill=black!30] (B) { \nodepart{second} marked = True};
\draw[-stealth] ([yshift=0.5cm]A) -- (B);
\node [data, right=1.0cm of B, fill=black!30] (C) { \nodepart{second} marked = True};
\draw[-stealth, transform canvas={yshift=0.25cm}] (B) -- (C);
\node [data, right=1.0cm of C, fill=black!30] (D) { \nodepart{second} marked = True};
\draw[-stealth, transform canvas={yshift=0.25cm}] (C) -- (D);
\draw[-stealth, transform canvas={yshift=0.25cm}] (D) to[out=5, in=20] (B);

\node [data, below=1cm of B, fill=white] (B1) { \nodepart{second} marked = False};
\node [data, right=1.0cm of B1, fill=white] (C1) { \nodepart{second} marked = False};
\draw[-stealth, transform canvas={yshift=0.25cm}] (B1) -- (C1);
\node [data, right=1.0cm of C1, fill=black!30] (D1) { \nodepart{second} marked = True};
\draw[-stealth, transform canvas={yshift=0.25cm}] (C1) -- (D1);

\draw[-stealth, transform canvas={yshift=0.25cm}] (B1) to[out=20, in=-170,distance=0.5cm] (C);
\draw[-stealth, transform canvas={yshift=0.25cm}] (C) to[out=0, in=-190,distance=0.5cm] (D1);
\node[below=0.6cm of C1] {Illustration after a mark.};



\end{tikzpicture}