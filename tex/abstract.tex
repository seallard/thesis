
\vspace*{3cm}

\begin{center}
	\textbf{Abstract}
\end{center}

{
  Novelty search is a technique used in evolutionary robotics to prevent search algorithms from prematurely
  converging on local optima. Instead of scoring solutions based on how good they are according to an objective
  function, novelty search assigns a score based on how different each solution is from previous ones. However, when the
  search space is large it no longer is useful to guide the search only by novelty. In this thesis, I propose
  two combinations of objective and novelty search. The first combination, called dynamic linearisation, uses
  a dynamic ratio of the novelty and objective score. The ratio is updated based on the performance of the search. The
  second combination, called novelty injection, alternates between using novelty and objective scores based on the performance of
  the search.
  The performance of the combinations is compared to novelty search, objective search and a static novelty ratio on a maze navigation task with
  three mazes of different complexity and size. Neither of the proposed combinations solved any of the mazes significantly
  faster than the static novelty ratio at the 0.05 level. However, the performance of dynamic linearisation was comparable
  to that of the static novelty ratio and might merit further investigation. For example, studying its performance when different
  constants it uses are varied.
}

\thispagestyle{empty}
\newpage