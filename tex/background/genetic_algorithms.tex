\subsection{Genetic algorithms}

Genetic algorithms are optimisation algorithms based on evolution. The fundamental idea is that
of repeated selection of solutions which are better according to some metric from a population (see Algorithm \ref{algorithm}).
Variation is introduced in the population by applying mutation- and crossover operators to the
selected solutions. The mutation operators slightly change the solutions whereas the crossover
operator is used to form offspring.

Each solution is represented by a genotype, also referred to as a genome or chromosome.
The mutation and crossover operators are applied to the genotype. During the evaluation, the
genotype is translated to its corresponding phenotype - the actual solution which can be evaluated.


\begin{algorithm}[H]
    \caption{Generic evolutionary algorithm}
    \begin{algorithmic}

\Procedure{GA}{}
    \State Initialise population
    \While{\textit{stopping condition not true}}
        \State Evaluation \Comment{Evaluate fitness of each individual}
        \State Selection  \Comment{Apply selection operator}
        \State Create new population \Comment{Apply crossover and mutation operators}
    \EndWhile
\EndProcedure

\end{algorithmic}
\label{algorithm}
\end{algorithm}
