
\subsection{Genetic Algorithms}
\todo{Add examples of applications?}
\newline
\todo{Explain settings which you probably will talk about in discussion}
\newline
Genetic algorithms (GAs) are search algorithms based on evolution. The implementation of GAs vary, but
generally follow the pattern outlined in Algorithm 1 below. Starting from a random population of 

GAs begin with
a random population of individuals. Each individual represents a solution to the problem being
studied. A fitness function scores how well each individual solves the problem. Variation is introduced 
in the population by letting individuals reproduce with each other, usually in proportion to their fitness 
scores. Offspring is created by applying by a crossover operator which combine parts of the parent solutions, 
much like the crossover between chromosomes during sexual reproduction in nature. A second source of variation
is applied in the form of mutations, each gene is randomly altered with some probability.


\begin{algorithm}[H]
    \caption{Generic genetic algorithm}
    \begin{algorithmic}

\Procedure{GA}{}
    \State Initialise population
    \While{\textit{stopping condition not true}}
        \State Evaluation \Comment{Evaluate fitness of each individual}
        \State Selection  \Comment{Apply selection operator}
        \State Create new population \Comment{Apply crossover and mutation operators}
    \EndWhile
\EndProcedure

\end{algorithmic}
\end{algorithm}

\todo{Add illustration of GAs}