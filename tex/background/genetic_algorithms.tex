\subsection{Genetic algorithms}
Genetic algorithms are optimisation algorithms based on evolution. They have been successfully
applied to problems in a wide array of domains, such as structural engineering \cite{engineering_gas}
and drug development \cite{drug_gas}.

The fundamental idea is that of repeated selection of solutions which are better according to
some metric from a population (see Algorithm \ref{algorithm}).
Variation is introduced in the population by applying mutation and crossover operators to the
selected solutions. The mutation operators slightly change the solutions whereas the crossover
operators are used to create new solutions by combining existing ones.

Each solution is encoded by a genotype, also referred to as a genome. The genome contains
genes which specify the solution. The crossover and mutation operators are applied to those genes.
During the evaluation, the genotype is translated to its corresponding phenotype -
the actual solution which can be evaluated. See \cite{compint} for more details.


\begin{algorithm}[H]
    \caption{Generic evolutionary algorithm}
    \begin{algorithmic}

\Procedure{GA}{}
    \State Initialise population
    \While{\textit{stopping condition not true}}
        \State Evaluation \COMMENT{Evaluate the fitness of each individual}
        \State Selection  \COMMENT{Apply the selection operator}
        \State Create new population \COMMENT{Apply the crossover and mutation operators}
    \EndWhile
\EndProcedure

\end{algorithmic}
\label{algorithm}
\end{algorithm}
