
\subsection{Genetic Algorithms}
\todo{Add examples of applications?}
\newline
\todo{Explain settings which you probably will talk about in discussion}
\newline
Genetic algorithms (GAs) are search algorithms based on evolution. GAs begin with
a random population of individuals. Each individual represents a solution to the problem being
studied. A fitness function scores how well each individual solves the problem. Variation is introduced 
in the population by letting individuals reproduce with each other, usually in proportion to their fitness 
scores. The offspring is formed by an operation called crossover which combine parts of the parent solutions, 
much like the crossover between chromosomes during sexual reproduction in nature. A second source of variation
is applied in the form of mutations, each gene is randomly altered with some probability.


\begin{algorithm}[H]
    \caption{Generic genetic algorithm}
    \begin{algorithmic}

\Procedure{GA}{}
    \State Initialise population
    \While{\textit{stopping condition not true}}
        \State Evaluation \Comment{Evaluate fitness of each individual}
        \State Selection  \Comment{Select individuals for reproduction}
        \State Reproduction \Comment{Apply crossover and mutation}
        \State Replace the old population with the new
    \EndWhile
\EndProcedure

\end{algorithmic}
\end{algorithm}

\todo{Add illustration of GAs}