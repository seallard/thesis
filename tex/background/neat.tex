\newpage

\subsection{Neuroevolution of Augmenting Topologies}
Neuroevolution uses evolutionary algorithms, such as genetic algorithms, to optimise ANNs \cite{neuroevolution_review}.
One such algorithm is Neuroevolution of Augmenting Topologies (NEAT), which evolves the topology and weights of ANNs \cite{neat_main}.
It gained traction when published since it adressed some previous problems in neuroevolution and illustrated that the learning speed could
be increased by evolving the structure of the networks.

NEAT uses a direct encoding; genomes contain link and node genes which specify the network. Each link gene references
the two node genes it span (see Figure).


\begin{figure}[htb]
    \resizebox{0.5\textwidth}{!}{\setlength{\arrayrulewidth}{1mm}
\setlength{\tabcolsep}{5pt}
\renewcommand{\arraystretch}{1}

%\newcolumntype{s}{>{\columncolor[HTML]{AAACED}} p{0.5cm}}

\arrayrulecolor[HTML]{DB5800}

\begin{tabular}{ |p{0.5cm}|p{0.5cm}|p{0.7cm}|p{1.5cm}|p{1.5cm}|  }
    \hline
    \multicolumn{5}{|c|}{Connection genes} \\

    \hline
    Id & In & Out & Weight & Enabled \\

    \hline

\end{tabular}}
\end{figure}
...
The innovation number attribute of the connection gene is used during crossover to align genes of the same origin.

\todo{Mutation operator figure}
\newline
\todo{Crossover figure}
\newline
\todo{Add illustration of NEAT genes here}
\newline
\todo{Add SUPER illustration of NEAT here}

