\newpage

\subsection{Neuroevolution of Augmenting Topologies}
Neuroevolution uses evolutionary algorithms, such as genetic algorithms, to optimise ANNs \cite{neuroevolution_review}.
One such algorithm is Neuroevolution of Augmenting Topologies (NEAT), which evolves the topology and weights of ANNs \cite{neat_main}.
It gained traction when published since it adressed some previous problems in neuroevolution and illustrated that the learning speed could
be increased by evolving the structure of the networks.

NEAT uses a direct encoding; genomes contain link and node genes which specify the network. Each link gene references
the two node genes it span (see Figure).


\begin{figure}[htb]
    \resizebox{0.5\textwidth}{!}{\begin{figure}[htb]
    \begin{mdframed}
        \begin{subfigure}[b]{0.45\textwidth}
            \centering
            \resizebox{1\textwidth}{!}{\input{resources/tex/genotype.tex}}
            \caption{Genotype.}
            \label{genotype}
        \end{subfigure}
        \begin{subfigure}[b]{0.45\textwidth}
            \centering
            \resizebox{0.65\textwidth}{!}{\input{resources/tex/phenotype.tex}}
            \caption{Phenotype.}
            \label{phenotype}
        \end{subfigure}
    \end{mdframed}
    \caption{Example of a genome and its corresponding ANN phenotype.}
    \label{mapping}
\end{figure}
}
\end{figure}
...
The innovation number attribute of the connection gene is used during crossover to align genes of the same origin.

\todo{Mutation operator figure}
\newline
\todo{Crossover figure}
\newline
\todo{Add illustration of NEAT genes here}
\newline
\todo{Add SUPER illustration of NEAT here}

