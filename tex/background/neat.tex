\newpage

\subsection{Neuroevolution of Augmenting Topologies}
Neuroevolution uses evolutionary algorithms, such as genetic algorithms, to optimise ANNs \cite{neuroevolution_review}.
One such algorithm is Neuroevolution of Augmenting Topologies (NEAT), which evolves the topology and weights of ANNs \cite{neat_main}.
It gained traction when published since it adressed some previous problems in neuroevolution and illustrated how the learning speed could
be increased by evolving the structure of the networks.

NEAT uses a direct encoding; genomes contain connection and node genes which specify the network. Each connection gene is identified
by a number and references the two node genes it spans, whether it is enabled and its weight (see Figure \ref{mapping}). The node gene
is also identified by a number and what type it is of.

\setlength{\arrayrulewidth}{1mm}
\setlength{\tabcolsep}{5pt}
\renewcommand{\arraystretch}{1}

%\newcolumntype{s}{>{\columncolor[HTML]{AAACED}} p{0.5cm}}

\arrayrulecolor[HTML]{DB5800}

\begin{tabular}{ |p{0.5cm}|p{0.5cm}|p{0.7cm}|p{1.5cm}|p{1.5cm}|  }
    \hline
    \multicolumn{5}{|c|}{Connection genes} \\

    \hline
    Id & In & Out & Weight & Enabled \\

    \hline

\end{tabular}

The initial population of genomes consists of structurally identical and fully connected networks without any hidden nodes.
New structures emerge by mutations which can add new node or connection genes to the genomes. Nodes are added by splitting an existing
connection, which is disabled and two new connection genes are added along with the new node gene (see Figure \ref{node_mutation}).

\begin{figure}[htb]
    \begin{mdframed}
        \begin{subfigure}[b]{0.45\textwidth}
            \centering
            \resizebox{1\textwidth}{!}{\setlength{\arrayrulewidth}{0.01mm}
\setlength{\tabcolsep}{5pt}
\renewcommand{\arraystretch}{1}

\arrayrulecolor[HTML]{000000}

\vspace{0.2cm}
\begin{tabular}{ |p{0.5cm}|p{1.8cm}|  }
    \hline
    \rowcolor{lightgray} \multicolumn{2}{|c|}{Node genes} \\
    \hline
    Id & Type \\
    \hline
    1 & \cellcolor{green!50} INPUT \\
    \hline
    2 & \cellcolor{green!50} INPUT \\
    \hline
    3 & \cellcolor{red!50} OUTPUT \\
    \hline
    4 & \cellcolor{blue!50} HIDDEN \\
    \hline
\end{tabular}

\begin{tabular}{ |p{0.5cm}|p{0.5cm}|p{0.7cm}|p{1.2cm}|p{1.4cm}|  }
    \hline
    \rowcolor{lightgray} \multicolumn{5}{|c|}{Connection genes} \\
    \hline
    Id & In & Out & Weight & Enabled \\
    \hline
    1 & \cellcolor{green!50} 1 & \cellcolor{red!50} 3 & 0.3 & True \\
    2 & \cellcolor{green!50} 2 & \cellcolor{red!50} 3 & 0.5 & False \\
    3 & \cellcolor{green!50} 2 & \cellcolor{blue!50} 4 & 1 & True \\
    4 & \cellcolor{blue!50} 4 & \cellcolor{red!50} 2 & 0.5 & True \\
    \hline
\end{tabular}
}
            \caption{Genotype.}
            \label{node_genotype}
        \end{subfigure}
        \begin{subfigure}[b]{0.45\textwidth}
            \centering
            \resizebox{0.65\textwidth}{!}{\def\layersep{2.5cm}

\begin{tikzpicture}[shorten >=1pt,->,draw=black!50, node distance=\layersep]
    \tikzstyle{every pin edge}=[<-,shorten <=1pt]
    \tikzstyle{neuron}=[circle,fill=black!25,minimum size=17pt,inner sep=0pt]
    \tikzstyle{input neuron}=[neuron, fill=green!50];
    \tikzstyle{output neuron}=[neuron, fill=red!50];
    \tikzstyle{hidden neuron}=[neuron, fill=blue!50];

    % Draw the input nodes.
    \foreach \name / \y in {1,...,2}
        \node[input neuron] (I-\name) at (0,-\y) {\y};

    %Draw the hidden node.
    \node[hidden neuron, right of=I-2] (H-1) {4};

    % Draw the output node.
    \node[output neuron, right of=H-2] (O-1) {3};

    % Connect input 1 with output node.
    \path (I-1) edge (O-1);
    \path (I-2) edge (H-1);
    \path (H-1) edge (O-1);

    \draw (I-1) -- (O-1) node [midway, fill=white] {1};
    \draw (I-2) -- (H-1) node [midway, fill=white] {3};
    \draw (H-1) -- (O-1) node [midway, fill=white] {4};

\end{tikzpicture}}
            \caption{Phenotype.}
            \label{node_phenotype}
        \end{subfigure}
    \end{mdframed}
    \caption{Example of a node mutation.}
    \label{node_mutation}
\end{figure}


Mutations can also add new links, which connect any previously unconnected nodes (see Figure \ref{link_mutation}).
Self loops and recurrent connections are allowed. A list of additional mutations employed by NEAT are outlined in Table X.

\begin{figure}[htb]
    \begin{mdframed}
        \begin{subfigure}[b]{0.45\textwidth}
            \centering
            \resizebox{1\textwidth}{!}{\setlength{\arrayrulewidth}{0.01mm}
\setlength{\tabcolsep}{5pt}
\renewcommand{\arraystretch}{1}

\arrayrulecolor[HTML]{000000}

\vspace{0.2cm}
\begin{tabular}{ |p{0.5cm}|p{1.8cm}|  }
    \hline
    \rowcolor{lightgray} \multicolumn{2}{|c|}{Node genes} \\
    \hline
    ID & Type \\
    \hline
    1 & \cellcolor{green!50} INPUT \\
    \hline
    2 & \cellcolor{green!50} INPUT \\
    \hline
    3 & \cellcolor{red!50} OUTPUT \\
    \hline
    4 & \cellcolor{blue!50} HIDDEN \\
    \hline
\end{tabular}

\begin{tabular}{ |p{0.5cm}|p{0.5cm}|p{0.7cm}|p{1.2cm}|p{1.4cm}|  }
    \hline
    \rowcolor{lightgray} \multicolumn{5}{|c|}{Connection genes} \\
    \hline
    ID & In & Out & Weight & Enabled \\
    \hline
    1 & \cellcolor{green!50} 1 & \cellcolor{red!50} 3 & 0.3 & True \\
    2 & \cellcolor{green!50} 2 & \cellcolor{red!50} 3 & 0.5 & False \\
    3 & \cellcolor{green!50} 2 & \cellcolor{blue!50} 4 & 1 & True \\
    4 & \cellcolor{blue!50} 4 & \cellcolor{red!50} 2 & 0.5 & True \\
    5 & \cellcolor{green!50} 1 & \cellcolor{blue!50} 4 & 0.4 & True \\
    \hline
\end{tabular}
}
            \caption{Genotype.}
            \label{link_genotype}
        \end{subfigure}
        \begin{subfigure}[b]{0.45\textwidth}
            \centering
            \resizebox{0.65\textwidth}{!}{\def\layersep{2.5cm}

\begin{tikzpicture}[shorten >=1pt,->,draw=black!50, node distance=\layersep]
    \tikzstyle{every pin edge}=[<-,shorten <=1pt]
    \tikzstyle{neuron}=[circle,fill=black!25,minimum size=17pt,inner sep=0pt]
    \tikzstyle{input neuron}=[neuron, fill=green!50];
    \tikzstyle{output neuron}=[neuron, fill=red!50];
    \tikzstyle{hidden neuron}=[neuron, fill=blue!50];

    % Draw the input nodes.
    \foreach \name / \y in {1,...,2}
        \node[input neuron] (I-\name) at (0,-\y) {\y};

    %Draw the hidden node.
    \node[hidden neuron, right of=I-2] (H-1) {4};

    % Draw the output node.
    \node[output neuron, right of=H-2] (O-1) {3};


    \draw (I-1) -- (O-1) node [midway, fill=white] {1};
    \draw (I-2) -- (H-1) node [midway, fill=white] {3};
    \draw (H-1) -- (O-1) node [midway, fill=white] {4};
    \draw (I-1) -- (H-1) node [midway, fill=white] {5};
\end{tikzpicture}}
            \caption{Phenotype.}
            \label{link_phenotype}
        \end{subfigure}
    \end{mdframed}
    \caption{ (a) The resulting genotype after a link mutation of the genotype from \ref{node_mutation}(a). (b) The
    corresponding network phenotype with a new forward link between nodes 1 and 4.}
    \label{link_mutation}
\end{figure}


New structures can also be created during crossover, in which the genes of two genomes are combined.
The id attribute of the genes, referred to as an innovation number, serves to uniquely identify each mutation that has occured.
A database mapping the innovation numbers with their corresponding structure is maintained. Each time a mutation occurs, the database
is referenced to check whether it has occured previously.
enabling identical structures to be labeled as such.
These numbers are used during speciation and crossover. NEAT divides the population of genomes into species based on their structural
similarity, which is measured as a compatibility score.


\todo{Explain speciation and fitness sharing}

\todo{Add SUPER illustration of NEAT here}
