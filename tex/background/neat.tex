\newpage

\subsection{Neuroevolution of Augmenting Topologies}
Neuroevolution uses evolutionary algorithms, such as genetic algorithms, to optimise ANNs \cite{neuroevolution_review}.
One such algorithm is Neuroevolution of Augmenting Topologies (NEAT), which evolves the topology and weights of ANNs \cite{neat_main}.
It gained traction when published since it adressed some previous problems in neuroevolution and illustrated that the learning speed could
be increased by evolving the structure of the networks.

NEAT uses a direct encoding; genomes contain link and node genes which specify the network. Each link gene references
the two node genes it span (see Figure \ref{mapping}).

\setlength{\arrayrulewidth}{1mm}
\setlength{\tabcolsep}{5pt}
\renewcommand{\arraystretch}{1}

%\newcolumntype{s}{>{\columncolor[HTML]{AAACED}} p{0.5cm}}

\arrayrulecolor[HTML]{DB5800}

\begin{tabular}{ |p{0.5cm}|p{0.5cm}|p{0.7cm}|p{1.5cm}|p{1.5cm}|  }
    \hline
    \multicolumn{5}{|c|}{Connection genes} \\

    \hline
    Id & In & Out & Weight & Enabled \\

    \hline

\end{tabular}

The initial population of genomes consists of structurally identical fully connected networks without any hidden nodes.
New structures emerge by mutations which can add new node or link genes to the genomes. New structures can also be created
during crossover, in which the genes of two genomes are combined.

\todo{Add illustration of mutations here}

\todo{Explain innovation numbers}
\newline
The id attribute of the genes, referred to as an innovation number, serves to uniquely identify each mutation that has occured.
A database mapping the innovation numbers with their corresponding structure is maintained, enabling identical structures to be labeled as such.
These numbers are used during speciation and crossover. NEAT divides the population of genomes into species based on their structural
similarity, which is measured as a compatibility score.



\todo{Explain crossover}
\newline


\todo{Explain speciation and fitness sharing}

\todo{Add SUPER illustration of NEAT here}
