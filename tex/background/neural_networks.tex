\subsection{Artificial Neural Networks}
Artificial Neural Networks (ANNs) are function approximators loosely inspired by nervous systems.
ANNs consist of nodes connected by directed links. Each node receives signals, either from
the environment or other nodes. Each link is associated with a weight which inhibits or excites the signal passing
over it. The nodes sum the incoming weighted signals, calculating a net input signal. A function
is applied to the net input signal to generate an output signal. The function is referred to as an activation function
and regulates the strength of the output signal given the net input signal. The signal is propagated through the network
until it reaches the final output node or nodes.

\todo{Add figure with illustration of artificial neuron}


%https://github.com/battlesnake/neural
%https://tex.stackexchange.com/questions/153957/drawing-neural-network-with-tikz

\label{Neural networks figure}

\begin{neuralnetwork}[height=4]
    \newcommand{\x}[2]{$x_#2$}
    \newcommand{\y}[2]{$\hat{y}_#2$}
    \newcommand{\hfirst}[2]{\small $h^{(1)}_#2$}
    \newcommand{\hsecond}[2]{\small $h^{(2)}_#2$}
    \inputlayer[count=3, bias=true, title=Input\\layer, text=\x]
    \hiddenlayer[count=4, bias=false, title=Hidden\\layer 1, text=\hfirst] \linklayers
    \outputlayer[count=2, title=Output\\layer, text=\y] \linklayers
\end{neuralnetwork}


Different learning algorithms are applied to update parameters of the network, such as the weights of the
links or the topology, as to improve the final output.

\todo{Describe activation functions}
These activation functions mimics the behaviour of biological neurons, where the neuron propagates
a signal only if the incoming signals exceed some threshold.

\todo{Describe bias neurons}

\todo{Describe different topologies}
Links that go backwards towards the inputs, so called recurrent links, gives the network short-term memory.

