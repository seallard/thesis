\subsection{Artificial Neural Networks}
Artificial Neural Networks (ANNs) are function approximators loosely inspired by nervous systems.
ANNs consist of nodes connected by weigthed directed links. Each node receives signals, either from
the environment or other nodes. The nodes sum the incoming weighted signals and applies an activation
function. The result of the activation function is forwarded as a signal to each of the connected nodes.

The weights of the links serves as

\todo{Add figure with illustration of artificial neuron}
\label{Neural networks figure}



Different learning algorithms are applied to update parameters of the network, such as the weights of the
links or the topology, as to improve the final output.

\todo{Describe activation functions}
These activation functions mimics the behaviour of biological neurons, where the neuron propagates
a signal only if the incoming signals exceed some threshold.

\todo{Describe bias neurons}

\todo{Describe different topologies}
Links that go backwards towards the inputs, so called recurrent links, gives the network short-term memory.

