\subsection{Novelty search}
Novelty search can be used with any evolutionary algorithm by replacing the fitness
function with a novelty metric. Each solution is characterised based on some aspect
of its behaviour. For example, in the domain of maze navigation,
the behaviour of each ANN can be characterised as the
final coordinates of the robot in the maze. A novelty metric specifies how distant a solution
is in the behaviour space compared to an archive of discovered solutions. It is a measure
of sparseness $\rho$ and can be computed at a point $x$ as the average distance of the
$k$-nearest neighbours in the archive
\[
    \rho(x) = \frac{1}{k} \sum_{i=0}^{k} dist(x, n_{i})
\]

where $n_{i}$ is the $i$th-nearest neighbour of $x$ using some distance metric $dist$ \cite{novelty_alone}.

It might seem like novelty search is an exhaustive search of the solution space. However,
depending on how the behaviour of the solutions is characterised, many different solutions
collapse to the same point in the behavioural space. Another advantage of novelty search
is that it follows the arrow of increasing complexity. Once all of the simple solutions
have been discovered, the only way to receive a higher score is to evolve a more complex
behaviour.