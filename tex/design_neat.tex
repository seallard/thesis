
\newpage
\section{Design and Implementation of NEAT}
\subsection{Overview}
I implemented NEAT in Python.

\subsection{Verification of functionality}
XOR is not linearly separable (see figure) which imposes some requirements on the ANN structures able to solve it (see figure).
As such, it is a suitable problem to use as a sanity check. Another advantage is that it provides a direct way to compare my
implementation with the original one, since they also used XOR and published their results \cite{neat_main}.

Each network was evaluated against the four input examples (see figure) and assigned a fitness according to the following expression

\begin{equation*} \label{eq:1}
    (4 - \sum_{i=1}^{4} expected_{i} - actual_{i})^2
\end{equation*}

where $expected_{i}$ is the correct answer for example $i$ and $actual_{i}$ is the actual output of the network for example $i$.

Data was collected from 100 runs. Each run was terminated once a network was found that solved
XOR or the number of networks evaluated exceeded 20 000. The parameters used can be found in
appendix \ref{parameter_values} and were largely based on the ones used in \cite{neat_phd}.

My results were largely consistent with the ones reported in \cite{neat_main}. The main difference
is that, while mine finds solutions in fewer evaluations, the networks are on average larger and
fewer minimal networks are found. The standard deviations are also slightly larger, indicating that
my implementation is more inconsistent in the networks it produces.


\todo{Add figure with: truth vals of XOR, plot of XOR and minimal network able to solve XOR (show weights!)}
\todo{Add histogram plot}