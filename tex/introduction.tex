\section{Introduction}

\todo{https://www.cs.cmu.edu/~jrs/sins.html}


\label{sec:intro}

Optimisation algorithms use objective functions to guide the search for optimal solutions.
But some problems and objective functions are deceptive, causing the algorithm to become stuck at
locally optimal solutions.

For example, consider the problem of optimising a controller for a bipedal
robot. Intuitively, one could base the objective function on the distance the robot is able to travel.
That would mean controllers which increase the distance the robot is able to travel are favoured.
However, this objective function might cause the algorithm to converge on unstable controllers which
propels the robot forwards without control. The solutions along the path to the optimal solution
do not necessarily resemble it. To be able to walk a long distance, it might be necessary to find
solutions for falling and crawling. The optimal solution is not always reachable by just
following the gradient of the objective function.

Many different techniques exists for avoiding and getting out of local optima. One such technique, called
novelty search, discards the notion of an objective. Instead, solutions are scored based on how novel
they are in comparison to previously found solutions.

In this thesis I investigate whether alternating between novelty and objective search provide any
benefits when solving a bipedal walking task. The bipedal walkers are controlled by neural networks
which are optimised using a genetic algorithm.

\todo{}


\subsection{Related work}
Several different techniques for combining novelty and fitness exists that perform better than novelty or
fitness based search alone on various tasks \cite{ns_study}. For example, novelty can be combined with fitness using a weighted sum \cite{}.
Another combination, called Minimal  uses the fitness scores to cull the population and...

Novelty search ensures the behaviour space is explored more uniformly whereas fitness exploits the discovered behaviours
by gradually improving them. This idea forms the basis of a type of evolutionary algorithm called quality diversity optimisation,
which generate repertoires of optimised diverse solutions.

