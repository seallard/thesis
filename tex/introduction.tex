\section{Introduction}

\label{sec:intro}

Population based optimisation algorithms use fitness functions to guide the search for optimal solutions.
But some problems and fitness functions are deceptive, causing the algorithms to become stuck at
locally optimal solutions.

For example, consider the problem of optimising a controller for a bipedal
robot. Intuitively, one could base the fitness function on the distance the robot is able to travel.
That would mean controllers which increase the distance the robot is able to travel are favoured.
However, this fitness function might cause the algorithm to converge on unstable controllers which
propels the robot forwards without control. The solutions along the path to the optimal solution
do not necessarily resemble it. To be able to walk a long distance, it might be necessary to find
solutions for falling and crawling. The optimal solution is not always reachable by following
the gradient of the fitness function.

Many different techniques exists for avoiding and getting out of local optima. One such technique used
in evolutionary robotics, called
novelty search, discards the notion of an objective. Instead, solutions are scored based on how different
they are in comparison to previously found solutions. But when the behaviour space becomes large, the performance
of novelty search degrades \cite{novelty_not_enough}. By combining novelty with objective search, diversity
can be maintained while discovered solutions are optimised.

In this thesis I compare two combinations of novelty and fitness and their performance
on a maze navigation task. The first combination, proposed in \cite{novelty_not_enough}, uses a weighted
sum of novelty and fitness in which the weight is updated based on the performance of the search.
The second combination simply alternates between fitness and novelty depending on the performance
of the search. The simulated maze-navigating robots are controlled by neural networks which are
optimised using a genetic algorithm.

\subsection{Related work}
Several combinations exists that perform better than novelty or fitness based search alone
on various tasks \cite{ns_study}. For example, novelty can be combined with fitness using a
fixed weighted sum \cite{novelty_not_enough}. Another class of combinations discards solutions below
some threshold fitness value \cite{minimal_ns}. It is also possible to simultaneously optimise
the novelty and fitness score \cite{multi_ns}.
