\section{Introduction}

\todo{https://www.cs.cmu.edu/~jrs/sins.html}


\label{sec:intro}

Evolutionary computation techniques to solve problems. Population of candidate solutions. Better are allowed to reproduce
and replace the previous population. Requires way of scoring the individual candidate solutions. Problem! Imagine we are
trying to evolve a controller for a robot with the goal of making it walk a long distance. One intuitive way of scoring the
candidate solutions would be based on the distance they travelled. That would enable solutions which made the robot travel
further to reproduce to a larger extent. But the fitness function can be deceptive. We might instead end up with robots which
uncontrollably lunges forwards in order to maximise the distance travelled. This is referred to deceptiveness and is common in more
complex problems. While the final goal is to evolve a robot which can locomote... The fitness landscape can be deceptive.
The stepping stones do not necessarily resemble the the final objective.

In this thesis I apply a neuroevolution algorithm to bipedal walking and compare its performance for two different fitness functions.

\todo{}


\subsection{Related work}
Several different techniques for combining novelty and fitness exists that perform better than novelty or
fitness based search alone on various tasks \cite{ns_study}. For example, novelty can be combined with fitness using a weighted sum \cite{}.
Another combination, called Minimal  uses the fitness scores to cull the population and...

Novelty search ensures the behaviour space is explored more uniformly whereas fitness exploits the discovered behaviours
by gradually improving them. This idea forms the basis of a type of evolutionary algorithms called quality diversity optimisation, which
generates repertoires of optimised diverse solutions.

